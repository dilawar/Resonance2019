\RequirePackage{luatex85,shellesc}
\documentclass[]{resonance}
\usepackage{pgf,tikz}
\usepackage{tikzsymbols}
\usepackage{nameref}
\usepackage{pgfplots}
\usepackage{grffile}
\usepackage{amsmath}
\usepackage{mathtools}
\usepackage{chemfig}
\usepackage{siunitx}
\usepackage{todonotes}
\usepackage{amssymb}
\usetikzlibrary{shapes,backgrounds,calc,arrows,arrows.meta}

\usepackage[acronym]{glossaries}
%\newglossaryentry{CaM}
%{
%    name=Calmodulin,
%    description={Calmodulin}
%}

\newacronym{camkii}{CaMKII}{calcium/calmodulin dependant protein Kinase II}
\newacronym{cam}{CaM}{calmodulin}
\newacronym{psd}{PSD}{Post Synaptic Densitiy}
\newacronym{ca}{Ca\textsuperscript{++}}{calcium}
\newacronym{pp1}{PP1}{protein phophatase 1}
\newacronym{pp2}{PP2}{protein phophatase 2}
\newacronym{cacam}{Ca\textsuperscript{++}/CaM}{calcium/calmodulin complex}
\newacronym{i1p}{I1P}{phosphorylated inhibitor-1}
\newacronym{i1ppp1}{I1P.PP1}{I1P-PP1 complex}
\newacronym{i1}{I1}{inhibitor-1}
\newacronym{can}{CaN}{calcineurin}
\newacronym{pka}{PKA}{protein kinase A}
\newacronym{darpp}{DARPP-32}{a dopamine- and cyclic-AMP regulated neuronal phosphoprotein}
\newacronym{i2}{I2}{inhibitor 2}
\newacronym{sbgn}{SBGN}{System Biology Graphical Notation}
\newacronym{ltp}{LTP}{Long Term Potentiation}
\newacronym{ltd}{LTD}{Long Term Depression}
\newacronym{nmda}{NMDA}{N-methyl-D-asparate}
\newacronym{nmdar}{NMDAR}{N-methyl-D-asparate receptor}
\newacronym{ampa}{AMPA}{$\alpha$-amino-3-hydroxy-5-methyl-4-isoxazolepropionic acid}
\newacronym{mz}{MZ}{Miller and Zhabotinksy}


\newcommand\Fig[1]{\textit{Figure~\ref{#1}}}
\newcommand\TT[1]{\texttt{#1}}

% Title Page
\title{Switches in the brain?} 
\secondTitle{A potential mechanism for long-term memory storage}
\author{Dilawar Singh}
\date{\today}
\begin{document}
\maketitle

% Author info here
\authorIntro{
    \includegraphics[width=2cm]{./dilawar.jpg}\\
    Dilawar Singh is currently a graduate student at National Center for
    Biological Sciences (NCBS), Bengaluru. His hobby is to convince people to
    move to open-source softwares to live happily ever after.
}


\begin{abstract}

    We forget often. But we remember some memories as long as we live.
    This means that our brain is capable of protecting memories for years.
    This is a remarkable feat given that the \emph{biochemical hardware}
    involved in forming biological memories is an extremely hostile place for
    its storage.  What are the challenges involved? And what kind of 
    biochemical mechanisms may overcome them? This article explores a major
    hypothesis that molecular switches may be behind our remarkable ability to
    remember for a lifetime.

\end{abstract}

\maketitle
\monthyear{May 2018}
\artNature{GENERAL  ARTICLE}

\section{Introduction}\label{sec:intro}

Our brain is made up of roughly 100 billion neurons, joined together with over
100 trillion connections called \textbf{synpase}s. Each neuron on average makes
1000 connections.  \leftHighlight{Each neuron, on average, receives roughly 1000
connections from other neurons.}It is now widely accepted that memories are
created by processes in the brain changing these connections. 

Lets label these synapses as $s_1, s_2, \ldots s_n$. A subset of these synapse
can participate in a memory formation, for example, my memory of being chased by
a ferocious street dog named \emph{Lalu} (lets call it $M_\text{Lalu}$) is
represented by the set of the synapses $M_\text{Lalu}=(s_{10}, s_{21},
s_{12},\ldots,s_{331})$ i.e. these connections were changed during my troubling
encounter with Lalu. I sometimes recall this memory whenever I see a similar
looking dog. 

The ability to learn quickly from an experience and to recall it has obvious
advantages. Learning is a form of memory (and so is confidence). We are going to
use these terms interchangeably. The ability to remember and recall painful
encounters with predators helps all animals in avoiding them.  The ability to
remember the location or cue of food, water and mates, seasonal events for
migration or reproduction are just a few other examples of usefulness of memory. 

\begin{figure}[!t] 
    \caption{Memory formation and forgetting. During formation of
    a memory, some synapses become stronger. The longer you can maintain these
    connections, the longer you can hold on to this memory.}
   \label{fig:engram}
   \includegraphics[width=\linewidth]{engram.pdf} 
\end{figure}

I can recall an experience as long as the set of synapses in which the
particular experience was stored remains \emph{intact}. Therefore our
ability to remember is contingent on our brain's ability to keep its connections
intact.  And on the other hand, our ability to learn is depends on our brain's
ability to change its connections. And here is the first challenge!

\subsection{Learning quickly v/s forgetting slowly, a zero-sum game}\label{subsec:zero_sum} 

For $M_\text{Lalu}$ (or any other memory) to remain intact, each of its
component ($s_{10}, s_{21}, s_{12} \ldots$) should also remain intact.  The
longer a synapse can keep itself unchanged, the better it will be at keeping the
memory. Lets assume that somehow I create a synapse which maintain its state for
a very long time (i.e. a rigid synapse). This synapse will not \emph{forget}
easily, but it causes another problem. Rigid synapse will not participate in any
memory formation anymore since learning requires change and it can't change. It
behaves like a read-only compact disk. On the other hand, if I create a synapse
which is easily changeable (i.e. a plastic synaspe), it will be good at learning
new experiences but won't be able to retain it for long. Plastic synapse forgets
easily.  We know that not only we remember for long time, we are capable of
learning quickly too. For example, honey bees can learn the location of food
after one encounter with flowers. \leftHighlight{Ability to change is good for
learning, but bad for remembering} Indeed a good memory system is the one which
learns quickly from new experiences and forgets old information as slowly as
possible.  Forgetting and remembering are the two sides of the same coin.  They
are conflicting demands -- a zero-sum game.  The challenge is to strike a
balance. 


% Hopefield network 
\section{Hopfield network -- associative memory network}\label{sec:hopfield}

% \begin{quote}
% In mathematical networks, synaptic plasticity is the only non-trivial element 
% available to produce interesting behaviour. If model networks can achieve anything 
% approaching the behaviour of their biological counterparts then it will be clear 
% that synaptic plasticity is remarkably powerful and likely to be of crucial importance. 
% On the other hand, if the mathematical models cannot approach biological complexity then 
% other elements such as more accurate descriptions of individual cell behaviour will 
% have to be included in the models until we learn what minimum set of behaviours is needed 
% to mimic biological systems. \\
% \hfill -- L F Abbott
% \end{quote}

Before we continue further, lets familiarize ourselves with one very popular
network which is made up of neuron like elements. In this network, patterns can
be stored and recalled. Memory storage and retrieval is trivially done by a
computer. It will be helpful to compare memory storage in the computer and the
brain. In the computer, we always know the address of every stored memory and we
access it by providing this address. The file icon on your desktop is a
graphical way of encoding this addressing scheme. This process is very similar
to looking up the index page in a reference book to find a chapter. Our brain,
on the other hand, is very unlikely to have such an indexing mechanism. 

We recall when we are provided with \textit{cues}. For example, when you see
some part of of a familiar person in a wedding album -- while the rest of the
person may be hidden behind others -- you could easily identify the person. And
many other memories of that person will also be recalled. A famous class of
recurrent neural network popularly known as Hopefield network can do just the
same as shown in \Fig{fig:hopfield}.

\begin{figure}[!hb]
    \centering
    \caption{Hopfield network with 100 spiking neurons. These \emph{recurrent} 
        configurations give rise to interesting brain-like
        computation. \textbf{(B)} 6 patterns (memory) i.e. NCBSXY are stored in this
        network. \textbf{(C)} When a very distorted \textit{cue} is applied to
        the network input, it \textit{fetches} one of stored pattern which is
        the \emph{closest} to the applied cue.
    }\label{fig:hopfield}
    \includegraphics[width=\linewidth]{./hopfield.pdf}
\end{figure}

How does this recurrent network work is beyond the scope of article. Readers are
encouraged to explore more by themselves. \emph{``How well we can explain biological
memory by these network''} is an active research area.  Though these networks are
extremely successful in accomplishing various \textit{brain-like} computation
(a. la. machine learning), we would like to advise the reader to be sceptical by
noting the following:

\begin{itemize}
    \item  Neurons used in these networks are highly simplified. \textit{Real}
        neurons are not this simple. Even though these simplified neurons
        capture the essential \textit{all-or-none} (electrical spike) way of
        communication and learning by changing synaptic connections, they do
        ignore rich local computations which can be accomplished by branches of
        these neurons (called \textit{dendrites}).
    \item  There is no strong evidence that neurons make such dense recurrent
        connections. However some studies have shown that Hopfield network can
        work with very sparse recurrent connections as well. 
    \item Activity in these networks does not match usually observed activity 
        in the primate brain during memory-recall experiments.
\end{itemize}

\leftHighlight{Solutions contributed by other disciplines are helpful for
comparison and contrast and often provide very useful insight. But in the end, these
solutions must be tested under the constraints imposed by biology.}

These network provide us with a framework to concretely think about the problem
of memory storage and its recall. We learn a great deal about these problem by
pointing out the limitations and failure of these models. Hopfield network has
properties which will sound very natural to us. Can you store as many memories
as you like in these networks? No. There is an upper limit. Adding more patterns
over maximum limit causes distortion in memories. When a cue is given, the
network no longer fetch the right pattern. It often fetches a pattern which was
not even stored; the retrieved pattern instead resemble some mixture of many
stored patterns. When too many memories are stored, they corrupt each other by
mixing up. If connections are allowed to decay in these networks, memories
starts disappearing with the weakest memory disappearing first.

After this necessary detour, lets go back to the main theme: how do synapses
maintain their state?

\section{How does a synapse maintain its state?}

Very complex biochemistry plays out during learning that changes the synapse.
Surprisingly, the net effect of this complex biochemistry can be summarised by a simple
mathematical expression. Ah, \emph{the unreasonable effectiveness of
mathematics}\cite{unreasonable_math}! Let's assume that synaptic strength $w$ is
tightly correlated with a chemical species $X$ found at synapse i.e. $w$ changes
with $X$.  The problem of maintenance of $w$ can be rephrased as the problem of
maintenance of the level -- or the activity -- of $X$. Therefore, the problem of
``\emph{synapse maintaining its state}'' becomes the problem of ``molecule
\emph{$X$ maintaining its state}'', a more concretely defined problem.

\begin{figure}[h!]
    \caption{Phosphorylation and deposphorylation of X. P is phosphatase.}\label{fig:model}
    \includegraphics[]{./fig_model.pdf}
\end{figure}

Lets assume that $X$ is converted to its \texttt{ACTIVE} form $X^*$  by adding a
phosphoryl group ($PO_4^{2-}$). The phosphoryl group is removed by a phosphatase
and $X^*$ is turned back into \texttt{INACTIVE} $X$. The phosphorylation and its
counterpart dephosphorylation are a very common motif for controlling various
chemical reactions by \textit{activating} and \textit{inactivating} protein
molecules. Once most if not all $X$ has been turned into $X^*$ during
memory formation, how do we make sure that $X^*$ does not turn back into $X$
(lose memory)?

\rightHighlight{Can you think of other set of hypotheses? It must conform to laws of chemistry!}
Lets mull over a solution to this problem of long term maintenance of $X^*$.
Lets propose that somehow following are true. 
\begin{enumerate}
    \item \textbf{(Amplification)} $X^*$ \textbf{auto-phosphorylate} itself i.e. \tikz[baseline]{ 
            \node (x) {$X$};
            \node[right=9mm of x] (xp) {$X^*$};
            \draw[-latex] (x) -- (xp);
            \draw[-latex] (xp) edge[out=120, in=90] ([xshift=7mm]x);
        }. If we manage to get sufficient $X^*$ somehow, it
        will act as a catalyst to its own production. $X^*$ will always remain
        high.
    \item Dephosphorylation of $X^*$ is minimized by controlling the number of
        $P$ or reducing the reaction rate.
\end{enumerate} 

Both (1) and (2) helps in making $X^*$ highly stable. Problem solved? No.  Now we
have constructed a very rigid synapse. Recall the \textit{rigid} v/s
\textit{pastic} synpase dilemma discussed previously (section
\ref{subsec:zero_sum}).
This synapse will definitely remember for longer long but it will be almost
impossible for this synapse to participate in any new learning.

As long as we are in the realm of theory, let's propose a solution to this
problem . We add another reaction say $P'+X^*\rightarrow P'X \rightarrow P'+X$
which deactivates $X^*$ when the \textit{need} arise. Phosphatase P' is different
than P. This adds another layer of control to an already complicated problem i.e.
forgetting is now controlled by another process. This requires one more
explanation: how does this new mechanism controlling \textit{forgetting} works?
And philosophically -- if you care about it -- it violates the principle of
\textbf{parsimony} which recommends to pick the simplest explanation.

We still have two big problems hiding underneath. We have not considered the
underlying biological hardware i.e. synapse in any detail where this biochemical
network suppose to function. The first problem is chemical noise 
. For biochemical system operating in very small volumes,
effect of chemical noise can be very strong. \leftHighlight{The volume of a
typical synapse is \SI{1e-20}{\cubic\meter}. At this volume, \SI{1}{\micro M}
concentration is roughly equal to 6 molecules.} There are over 200 types of
protein molecules in a typical synapse. Indeed, most of these protein molecules have few
tens of molecules.  The brain is always active and  the chemical noise caused by
the background activity in the brain will surely turn some molecules of $X$ into
$X^*$. Then due to auto-phosphorylation, sooner than later, all of the $X$ will be turned
into $X^*$. We have created a very stable memory of nothing but background
noise. This is highly undesirable!

The second problem is \textit{turnover} i.e. old molecules are constantly degraded
and being replaced by newly minted molecules. Assume that at the time of memory
formation, we had 100 molecules of $X^*$ in synapse. And also assume that on
average, every day one new molecule (i.e. $X$) replaces an old one ($X^*$).
After 50 days, half of the synaptic strength is gone! We must have a
\textit{refresh} mechanism by which we make sure that the new molecule quickly
changes its state according to the state of synapse i.e. newborn $X$ becomes
$X^*$ most molecules at synapse are $X^*$.

Effectively, we want a stable \texttt{ACTIVE} state (where all $X$ are $X^*$)
and don't want chemical noise to turn $X$ into $X^*$. We want a switch like
behaviour -- if it is \textit{OFF} or \text{ON} it tends to stay \textit{OFF} or
\texttt{ON} respectively. A significant force is required to flip a switch; it
does not get flipped by noise. If few $X$ are turned into $X^*$ by background
noise, we expect them to be quickly turned back into $X$ by phosphatase. And if
during memory formation, a significant portion of $X$ has been turned into $X^*$
then we expect that any $X^*$ turned back into $X$ is quickly activated again
into $X^*$.  This system should operate like a switch which does not flip unless
significant force is applied. These are called \textbf{bistable switch}es. 

\begin{figure}[t!]
\centering
\includegraphics[width=\linewidth]{./fig_model_b.pdf}
\caption{A hypothetical network which can solve the problem of chemical noise and
    turnover with suitable parameters (how?). The activation step is divided into slow and fast
    components such that fluctuations caused by background noise do not cause system to 
    activate itself. $X^*$ also partially activate $X$ to $X^\sim$ to overcome \textit{turnover}. 
}\label{fig:model_bistable}
\end{figure}


Is there any  proof that bistable systems are even possible? Do they occur at
all in living cells?  Bistability (and its close relative oscillations) are very
common in biology; from cellular level to population levels. So if it won't be
surprising if we find bistable switches in syanpses as well.  \emph{Is there a set
of chemical reactions which forms a bistable switch at synapse?} Various studies
have shown that \gls{camkii} may form a bistable switch in synapse.

\section{Molecular bistable switch at synapse}\label{sec:molecular_switch}

\begin{figure}[ht!]
    \centering
    \includegraphics[width=8cm]{./lisman_bistable.pdf}
    \caption{Reaction in a bistable switch proposed by John Lisman. Modified
        from \cite{lisman1985}. Permission not required for reuse for
        educational purpose.
    }\label{fig:lisman}
\end{figure}

John Lisman hypothesised that a kinase and a phosphatase together
(\Fig{fig:lisman}) can form a bistable switch which is stable against
\emph{turnover}. \gls{camkii} and its phosphatase \gls{pp1} were identified as
the hypothesised kinase and phosphatase. This chemical system had been
extensively studied using computational models for over two decades
\cite{sandstorm}. There is evidence that \gls{camkii} is bistable \emph{in
vitro} conditions. 

\gls{camkii} is known to  play an important role in memory formation.  In
experiments involving mice, deactivating \gls{camkii} in any way has always
resulted in impairment of memory formation and learning. \gls{camkii} molecule
also has many interesting properties which makes it an attractive candidate for
storing memory. 12 to 14 subunits of \gls{camkii} make up one molecules, usually
arranged in dodecameric (top view \tikz{  \foreach \i in {1,2,...,6} {
        \node[circle, fill=blue!50, shift=(60*\i:1mm),inner sep=1pt] (b\i) at (0,0) {};
        } }) or tetradecameric form (top view \tikz{  \foreach \i in {1,2,...,7} {
        \node[circle, fill=blue!50, shift=(360*\i/7:1mm),inner sep=1pt] (b\i) at (0,0) {};
} }) . Activation of its first subunit is very slow. Once
a subunit has been activated, it catalyses activation of its neighbours i.e.
\gls{camkii} auto-phosphorylates itself. Moreover fully active \gls{camkii}
holoenzyme can loose an \textit{ACTIVE} subunit which can be picked up by
another holoenzyme. This process is called subunit-exchange and this is a way by
which \gls{camkii} spreads its activation (\Fig{fig:camkii_summary}).

\leftHighlight{
    \includegraphics[width=1.2cm]{./camkii_pdb.png}
    Top view of dodecameric \gls{camkii} (12 subunits). From Protein Data
    Bank (https://www.rcbs.org).
}

In our computational study of this pathway, we show that subunit-exchange
improves information retention capacity of \gls{camkii}. We also show that
distributed clusters of \gls{camkii} can form very stable bistable switches. And
it also operate as integrator which is often observed in experiments. In short,
we show the subunit-exchange makes \gls{camkii} molecule better at retaining
information and it is likely that \gls{camkii} is bistable in special
micro-environments. To prove it, one needs to observe single molecule activity
in synapse near surface of synapse which is a very challenging experiment.

\begin{figure}[h!]
    \caption{ \textbf{(A)} Graphical representation of \gls{camkii} signalling
        pathway. \textbf{(B)} This pathway shows bistable behaviour when synapse
        is receiving background activity. One trajectory is shown for system with 
        15 molecules. \textbf{(C)} Stability of synapse increases exponentially
        with number of \gls{camkii} molecules. With $\sim 60$ molecules, switch
        remain stable for roughly $150$ years.
    }\label{fig:camkii_summary}
    \centering
    \includegraphics[width=\linewidth]{./resonance_camkii.pdf}
\end{figure}


% References section
\section{Conclusion}

In this article, we have discussed why bistable motif is an attractive candidate
for storing biological memories. Most support for this idea has came from
computational studies. To really prove it, we need experimental data supporting
this hypothesis. There is growing experimental evidence that synapse change in
\textit{all-or-none} manner, a finding which is consistent with this idea. Some
studies claim that changes are graded i.e. synapse changes in step-wise manner
much like a \textbf{multi-stable} synapse. A multi-stable synapse is an ensemble
of many bistable components. Whether \gls{camkii} is bistable in synapse (or in
some part of it) is still an open question. So far there is no concrete evidence
that it is. There could be other still unknown mechanisms which can give rise to
bistability. Given that bistable chemical motifs are widespread in cells, it is
reasonable to believe that there are indeed switches in our brain -- much like
flip-flops in your pen-drives and memory cards -- evolved to keep our memories
safe from the onslaught of time and noise.

\section*{Acknowledgements}
I'd like to thank Somya Mani for helpful suggestions on the manuscript.


\begin{thebibliography}{99} 
    \bibitem{lisman1985} 
    Lisman J. E., 
    \textit{A mechanism for memory storage insensitive to molecular turnover: a
    bistable autophosphorylating kinase}. 
    Proc. Natl. Acad. Sci. USA, May 1985

    \bibitem{koch1999}
    Christof Koch
    \textit{Biophysics of computations}.
    Oxford University Press, 1999.

    \bibitem{sandstorm} 
    Malin Sandstorm,
    \textit{Models of CaMKII activation},
    Master Thesis, Royal Institute Of Technology Sweden 

    \bibitem{unreasonable_math}
    Eugene Wigner,
    \textit{The Unreasonable Effectiveness of Mathematics in the Natural Sciences},
     Communications in Pure and Applied Mathematics, vol. 13, No. I (February 1960)

\end{thebibliography}
\correspond{
    Bhalla Lab,
    National Center for Biological Sciences, Bengaluru \\
    GKVK Campus, Bellary Road \\
    Bengaluru - 560065.  \\
    Email: dilawars@ncbs.res.in
}

\end{document}
